% Options for packages loaded elsewhere
\PassOptionsToPackage{unicode}{hyperref}
\PassOptionsToPackage{hyphens}{url}
%
\documentclass[
]{book}
\usepackage{lmodern}
\usepackage{amssymb,amsmath}
\usepackage{ifxetex,ifluatex}
\ifnum 0\ifxetex 1\fi\ifluatex 1\fi=0 % if pdftex
  \usepackage[T1]{fontenc}
  \usepackage[utf8]{inputenc}
  \usepackage{textcomp} % provide euro and other symbols
\else % if luatex or xetex
  \usepackage{unicode-math}
  \defaultfontfeatures{Scale=MatchLowercase}
  \defaultfontfeatures[\rmfamily]{Ligatures=TeX,Scale=1}
\fi
% Use upquote if available, for straight quotes in verbatim environments
\IfFileExists{upquote.sty}{\usepackage{upquote}}{}
\IfFileExists{microtype.sty}{% use microtype if available
  \usepackage[]{microtype}
  \UseMicrotypeSet[protrusion]{basicmath} % disable protrusion for tt fonts
}{}
\makeatletter
\@ifundefined{KOMAClassName}{% if non-KOMA class
  \IfFileExists{parskip.sty}{%
    \usepackage{parskip}
  }{% else
    \setlength{\parindent}{0pt}
    \setlength{\parskip}{6pt plus 2pt minus 1pt}}
}{% if KOMA class
  \KOMAoptions{parskip=half}}
\makeatother
\usepackage{xcolor}
\IfFileExists{xurl.sty}{\usepackage{xurl}}{} % add URL line breaks if available
\IfFileExists{bookmark.sty}{\usepackage{bookmark}}{\usepackage{hyperref}}
\hypersetup{
  pdftitle={Curso Pandas 1.x en español},
  pdfauthor={Kevin Farinango (Asuskf)},
  hidelinks,
  pdfcreator={LaTeX via pandoc}}
\urlstyle{same} % disable monospaced font for URLs
\usepackage{color}
\usepackage{fancyvrb}
\newcommand{\VerbBar}{|}
\newcommand{\VERB}{\Verb[commandchars=\\\{\}]}
\DefineVerbatimEnvironment{Highlighting}{Verbatim}{commandchars=\\\{\}}
% Add ',fontsize=\small' for more characters per line
\usepackage{framed}
\definecolor{shadecolor}{RGB}{248,248,248}
\newenvironment{Shaded}{\begin{snugshade}}{\end{snugshade}}
\newcommand{\AlertTok}[1]{\textcolor[rgb]{0.94,0.16,0.16}{#1}}
\newcommand{\AnnotationTok}[1]{\textcolor[rgb]{0.56,0.35,0.01}{\textbf{\textit{#1}}}}
\newcommand{\AttributeTok}[1]{\textcolor[rgb]{0.77,0.63,0.00}{#1}}
\newcommand{\BaseNTok}[1]{\textcolor[rgb]{0.00,0.00,0.81}{#1}}
\newcommand{\BuiltInTok}[1]{#1}
\newcommand{\CharTok}[1]{\textcolor[rgb]{0.31,0.60,0.02}{#1}}
\newcommand{\CommentTok}[1]{\textcolor[rgb]{0.56,0.35,0.01}{\textit{#1}}}
\newcommand{\CommentVarTok}[1]{\textcolor[rgb]{0.56,0.35,0.01}{\textbf{\textit{#1}}}}
\newcommand{\ConstantTok}[1]{\textcolor[rgb]{0.00,0.00,0.00}{#1}}
\newcommand{\ControlFlowTok}[1]{\textcolor[rgb]{0.13,0.29,0.53}{\textbf{#1}}}
\newcommand{\DataTypeTok}[1]{\textcolor[rgb]{0.13,0.29,0.53}{#1}}
\newcommand{\DecValTok}[1]{\textcolor[rgb]{0.00,0.00,0.81}{#1}}
\newcommand{\DocumentationTok}[1]{\textcolor[rgb]{0.56,0.35,0.01}{\textbf{\textit{#1}}}}
\newcommand{\ErrorTok}[1]{\textcolor[rgb]{0.64,0.00,0.00}{\textbf{#1}}}
\newcommand{\ExtensionTok}[1]{#1}
\newcommand{\FloatTok}[1]{\textcolor[rgb]{0.00,0.00,0.81}{#1}}
\newcommand{\FunctionTok}[1]{\textcolor[rgb]{0.00,0.00,0.00}{#1}}
\newcommand{\ImportTok}[1]{#1}
\newcommand{\InformationTok}[1]{\textcolor[rgb]{0.56,0.35,0.01}{\textbf{\textit{#1}}}}
\newcommand{\KeywordTok}[1]{\textcolor[rgb]{0.13,0.29,0.53}{\textbf{#1}}}
\newcommand{\NormalTok}[1]{#1}
\newcommand{\OperatorTok}[1]{\textcolor[rgb]{0.81,0.36,0.00}{\textbf{#1}}}
\newcommand{\OtherTok}[1]{\textcolor[rgb]{0.56,0.35,0.01}{#1}}
\newcommand{\PreprocessorTok}[1]{\textcolor[rgb]{0.56,0.35,0.01}{\textit{#1}}}
\newcommand{\RegionMarkerTok}[1]{#1}
\newcommand{\SpecialCharTok}[1]{\textcolor[rgb]{0.00,0.00,0.00}{#1}}
\newcommand{\SpecialStringTok}[1]{\textcolor[rgb]{0.31,0.60,0.02}{#1}}
\newcommand{\StringTok}[1]{\textcolor[rgb]{0.31,0.60,0.02}{#1}}
\newcommand{\VariableTok}[1]{\textcolor[rgb]{0.00,0.00,0.00}{#1}}
\newcommand{\VerbatimStringTok}[1]{\textcolor[rgb]{0.31,0.60,0.02}{#1}}
\newcommand{\WarningTok}[1]{\textcolor[rgb]{0.56,0.35,0.01}{\textbf{\textit{#1}}}}
\usepackage{longtable,booktabs}
% Correct order of tables after \paragraph or \subparagraph
\usepackage{etoolbox}
\makeatletter
\patchcmd\longtable{\par}{\if@noskipsec\mbox{}\fi\par}{}{}
\makeatother
% Allow footnotes in longtable head/foot
\IfFileExists{footnotehyper.sty}{\usepackage{footnotehyper}}{\usepackage{footnote}}
\makesavenoteenv{longtable}
\usepackage{graphicx,grffile}
\makeatletter
\def\maxwidth{\ifdim\Gin@nat@width>\linewidth\linewidth\else\Gin@nat@width\fi}
\def\maxheight{\ifdim\Gin@nat@height>\textheight\textheight\else\Gin@nat@height\fi}
\makeatother
% Scale images if necessary, so that they will not overflow the page
% margins by default, and it is still possible to overwrite the defaults
% using explicit options in \includegraphics[width, height, ...]{}
\setkeys{Gin}{width=\maxwidth,height=\maxheight,keepaspectratio}
% Set default figure placement to htbp
\makeatletter
\def\fps@figure{htbp}
\makeatother
\setlength{\emergencystretch}{3em} % prevent overfull lines
\providecommand{\tightlist}{%
  \setlength{\itemsep}{0pt}\setlength{\parskip}{0pt}}
\setcounter{secnumdepth}{5}
\usepackage{booktabs}
\usepackage[]{natbib}
\bibliographystyle{apalike}

\title{Curso Pandas 1.x en español}
\author{Kevin Farinango (Asuskf)}
\date{2020-08-26}

\begin{document}
\maketitle

{
\setcounter{tocdepth}{1}
\tableofcontents
}
\hypertarget{introducciuxf3n}{%
\chapter{Introducción}\label{introducciuxf3n}}

Pandas es una librería para creación y manipulación de datos con Python. Está librería manipula datos tabulares en filas y columnas cómo encontrarías en una base de datos o en una hoja de cálculo. Pandas es aprovechado por: \textbf{Científicos de datos, analistas, programadores, ingenieros que lo usan para moldear sus datos}.

Pandas se limita a \textbf{``datos pequeños''} (datos que pueden caber en la memoria de una sola máquina). Sin embargo, la sintaxis y las operaciones se han adoptado o inspirado a otros proyectos: PySpark, Dask, Modin, cuDF, Baloo, Dexplo, Tabel, StaticFrame, entre otros.

\hypertarget{para-quiuxe9n-es-esta-guia}{%
\section{Para quién es esta guia}\label{para-quiuxe9n-es-esta-guia}}

Esta guía fue pensada en principiantes de habla hispana que estén empezando en el mundo de la ciencia de datos. Esta guía tiene la intención de dotar al principiante de los conocimientos necesarios para manipular y analizar sus datos con una herramienta potente y de fácil uso como es Pandas.

La presente guía tiene más de 50 ejercicios que van de básico a intermedio con los cuales el lector podrá poner en práctica e interiorizar lo aprendido. Para mejor comprensión cada ejercicio se detalla pasos a paso.

\hypertarget{quuxe9-cubre-esta-guuxeda}{%
\section{Qué cubre esta guía}\label{quuxe9-cubre-esta-guuxeda}}

\begin{itemize}
\item
  \textbf{Capítulo 1, Introducción:} Decripción de lo que tiene el libro y para quien va dirigido.
\item
  \textbf{Capítulo 2, Fundamentos de Pandas:} Presenta la anatomía y el vocabulario utilizado para identificar las dos estructuras de datos principales de pandas, la Serie y el DataFrame con sus respectivos métodos.
\item
  \textbf{Capítulo 3, Operaciones esenciales enun Dataframe:} Se centra en las operaciones comunes que se realizan durante el análisis de datos.
\item
  \textbf{Capítulo 4, Importación y creación de Dataframes:} Analiza las diversas formas de importar datos y crea DataFrames.
\item
  \textbf{Capítulo 5, Empezando con la ciencia de datos:} Presenta técnicas de análisis básicas que sirven para comparar números y datos categóricos. Este capítulo también demostrará técnicas de visualización comunes.
\item
  \textbf{Capítulo 6, Trabajando con subconjuntos:} Cubre las muchas formas variadas y potencialmente confusas de seleccionar diferentes subconjuntos de datos.
\item
  \textbf{Capítulo 7, Filtrando los subconjutnos:} Explica varias formas de cómo seleccionar diferentes subconjuntos de datos.
\item
  \textbf{Capítulo 8, Consultas booleanas:} Consultas basadas en condiciones booleanas
\item
  \textbf{Capítulo 9, el uso de index en Pandas:} El uso correcto de index para evitar resultados erróneos.
\item
  \textbf{Capítulo 10, Agrupación:} Presenta métodos de agrupación y la construcción de funciones personalizadas para aplicar a diferentes grupos
\item
  \textbf{Capítulo 11, Ordenando el dataframe:} Explica qué son los datos ordenados, por qué son tan importantes y que métodos pueden ser usados para transformar conjuntos de datos desordenados en ordenados.
\item
  \textbf{Capítulo 12, Combinando objetos Pandas:} Presenta varios métodos para combinar DataFrames y Series vertical u horizontalmente.
\item
  \textbf{Capítulo 13, Trabajando con las series de tiempo:} Presenta distintos métodos para trabajar con las series de tiempo en cualquier dimensión de tiempo posible
\item
  \textbf{Capítulo 14, Visualización de datos:} Tips para crear visualizaciones estéticamente agradables usando: Pandas, Seaborn y Matplotlib
\item
  \textbf{Capítulo 15, Test del dataframe:}Explora los mecanismos de test de los DataFrames y los resultados que presenta Pandas.
\end{itemize}

\hypertarget{intro}{%
\chapter{Fundamentos de Pandas}\label{intro}}

You can label chapter and section titles using \texttt{\{\#label\}} after them, e.g., we can reference Chapter \ref{intro}. If you do not manually label them, there will be automatic labels anyway, e.g., Chapter \ref{methods}.

Figures and tables with captions will be placed in \texttt{figure} and \texttt{table} environments, respectively.

\begin{Shaded}
\begin{Highlighting}[]
\KeywordTok{par}\NormalTok{(}\DataTypeTok{mar =} \KeywordTok{c}\NormalTok{(}\DecValTok{4}\NormalTok{, }\DecValTok{4}\NormalTok{, }\FloatTok{.1}\NormalTok{, }\FloatTok{.1}\NormalTok{))}
\KeywordTok{plot}\NormalTok{(pressure, }\DataTypeTok{type =} \StringTok{'b'}\NormalTok{, }\DataTypeTok{pch =} \DecValTok{19}\NormalTok{)}
\end{Highlighting}
\end{Shaded}

\begin{figure}

{\centering \includegraphics[width=0.8\linewidth]{Fundamentos-de-Pandas-Bookdown_files/figure-latex/nice-fig-1} 

}

\caption{Here is a nice figure!}\label{fig:nice-fig}
\end{figure}

Reference a figure by its code chunk label with the \texttt{fig:} prefix, e.g., see Figure \ref{fig:nice-fig}. Similarly, you can reference tables generated from \texttt{knitr::kable()}, e.g., see Table \ref{tab:nice-tab}.

\begin{Shaded}
\begin{Highlighting}[]
\NormalTok{knitr}\OperatorTok{::}\KeywordTok{kable}\NormalTok{(}
  \KeywordTok{head}\NormalTok{(iris, }\DecValTok{20}\NormalTok{), }\DataTypeTok{caption =} \StringTok{'Here is a nice table!'}\NormalTok{,}
  \DataTypeTok{booktabs =} \OtherTok{TRUE}
\NormalTok{)}
\end{Highlighting}
\end{Shaded}

\begin{table}

\caption{\label{tab:nice-tab}Here is a nice table!}
\centering
\begin{tabular}[t]{rrrrl}
\toprule
Sepal.Length & Sepal.Width & Petal.Length & Petal.Width & Species\\
\midrule
5.1 & 3.5 & 1.4 & 0.2 & setosa\\
4.9 & 3.0 & 1.4 & 0.2 & setosa\\
4.7 & 3.2 & 1.3 & 0.2 & setosa\\
4.6 & 3.1 & 1.5 & 0.2 & setosa\\
5.0 & 3.6 & 1.4 & 0.2 & setosa\\
\addlinespace
5.4 & 3.9 & 1.7 & 0.4 & setosa\\
4.6 & 3.4 & 1.4 & 0.3 & setosa\\
5.0 & 3.4 & 1.5 & 0.2 & setosa\\
4.4 & 2.9 & 1.4 & 0.2 & setosa\\
4.9 & 3.1 & 1.5 & 0.1 & setosa\\
\addlinespace
5.4 & 3.7 & 1.5 & 0.2 & setosa\\
4.8 & 3.4 & 1.6 & 0.2 & setosa\\
4.8 & 3.0 & 1.4 & 0.1 & setosa\\
4.3 & 3.0 & 1.1 & 0.1 & setosa\\
5.8 & 4.0 & 1.2 & 0.2 & setosa\\
\addlinespace
5.7 & 4.4 & 1.5 & 0.4 & setosa\\
5.4 & 3.9 & 1.3 & 0.4 & setosa\\
5.1 & 3.5 & 1.4 & 0.3 & setosa\\
5.7 & 3.8 & 1.7 & 0.3 & setosa\\
5.1 & 3.8 & 1.5 & 0.3 & setosa\\
\bottomrule
\end{tabular}
\end{table}

You can write citations, too. For example, we are using the \textbf{bookdown} package \citep{R-bookdown} in this sample book, which was built on top of R Markdown and \textbf{knitr} \citep{xie2015}.

\hypertarget{literature}{%
\chapter{Literature}\label{literature}}

Here is a review of existing methods.

\hypertarget{creaciuxf3n-y-persistencia-de-un-dataframe}{%
\chapter{Creación y persistencia de un Dataframe}\label{creaciuxf3n-y-persistencia-de-un-dataframe}}

We describe our methods in this chapter.

\hypertarget{empezando-con-la-ciencia-de-datos}{%
\chapter{Empezando con la ciencia de datos}\label{empezando-con-la-ciencia-de-datos}}

Some \emph{significant} applications are demonstrated in this chapter.

\hypertarget{python}{%
\section{Python}\label{python}}

\hypertarget{explorando-el-dataframe}{%
\chapter{Explorando el Dataframe}\label{explorando-el-dataframe}}

We have finished a nice book.

  \bibliography{book.bib,packages.bib}

\end{document}
